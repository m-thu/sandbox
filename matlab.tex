\documentclass[a4paper,fontsize=11pt,parskip]{scrartcl}
\usepackage[utf8]{inputenc}
\usepackage[ngerman]{babel}
%\usepackage[T1]{fontenc}
\usepackage{hyperref}
\usepackage{acronym}
\usepackage{graphicx}
\usepackage{subcaption}
\usepackage{amsmath, amssymb}
\usepackage{tikz}
\usepackage[european]{circuitikz}
%
%\usepackage[osf]{libertine}
%\usepackage{zi4} % Inconsolata
%\usepackage[libertine,cmbraces]{newtxmath}
%
\usepackage{booktabs}
\usepackage{makecell}
\usepackage{nicefrac}
\usepackage{upgreek}
%
\usepackage{pdfpages}
\usepackage{listings}
\lstset{numbers=left,
	frame=none,
	numberstyle=\tiny,
	basicstyle=\footnotesize,
	showstringspaces=false,
	%keywordstyle=\color{blue},
	%commentstyle=\em\color{gray},
	tabsize=3,
	numbersep=5pt,
	%morecomment=[s][\color{blue}]{<<<}{>>>},
	%morekeywords={float3,float4,__device__,__global__,__shared__,__constant__,threadIdx,blockIdx,blockDim,gridDim,\_\_syncthreads}}
	language=matlab,
	morekeywords={switch,case,otherwise}
}
%
%
%
%
\usepackage{geometry}
\geometry{left=0.5cm,right=0.5cm,top=0.5cm,bottom=0.5cm}
\usepackage{trfsigns}
\begin{document}
{\Huge Matlab/Simulink Cheat Sheet}\\
{\large Matthias Thumann}\\
{\large https://github.com/m-thu/sandbox/blob/master/matlab.tex}


{\Large Graphics}


\begin{lstlisting}
clear      % Clear all variables
close all  % Close all figures
format     % Reset output format for numbers to default

figure(n); % Select figure n

plot(y);            % Plots all values in y as a function of the index
plot(x, y);         % Plots all (x,y) pairs
axis([x1 x2 y1 y2]; % Specify plotting area

title('text');            % Plot title
xlabel('text');           % Label x-axis
ylabel('text');           % Label y-axis
grid;                     % Enable grid
legend('text1', 'text2'); % Legend in the order of the plotted functions

subplot(m, n, p); % m rows, n columns, p: number of subplot (numbered left to right,
                  % top to bottom)

hold on;  % Keep graphics in current figure
hold off; % Reset hold functionality
\end{lstlisting}


{\Large Control Flow Constructs}

If ... else:
\begin{lstlisting}
if a == 0
   b = 2;
elseif a < 0
   b = -a;
else
   b = a;
end;
\end{lstlisting}

Switch ... case:
\begin{lstlisting}
switch (x)
   case (1)
      disp('x = 1')
   case (2)
      disp('x = 2')
   otherwise
      disp('default')
end;
\end{lstlisting}

For-loop:
\begin{lstlisting}
for k = 1:100
   u(k) = k^2;
   if k == 9
      break;
end;
\end{lstlisting}

While-loop:
\begin{lstlisting}
z = 0;
while z <= 10
   z = z + 1;
end;
\end{lstlisting}

{\Large Analysis of AC Circuits}

Resistance $R$: $\underline Z_R = R$, $u_R = R\cdot i$

Inductance: $L$: $\underline Z_L = \mathrm{j}\omega L=sL$, $u_L = L\cdot\frac{\mathrm{d}\,i_L}{\mathrm{d}\,t}$

Capacity $C$: $\underline Z_C = \frac{1}{\mathrm{j}\omega C}=\frac{1}{sC}$, $i_C = C\cdot\frac{\mathrm{d}\,u_C}{\mathrm{d}\,t}$

Impedance of the circuit: $\underline Z = \dots$, total current: $\underline I = \frac{\underline U}{\underline Z}$, voltages at the components:
\begin{lstlisting}
omega = 2*pi * f;
Z_C   = 1 / (1j * omega * C);
Z_L   = 1j * omega * L;
Z     = R + Z_C + Z_L;

i   = U / Z; % U: Amplitude
U_R = i * R;
U_C = i * Z_C;
U_L = i * Z_L;
\end{lstlisting}

Plotting voltages in the time domain:
\begin{lstlisting}
T = 1 / f;
t = 0:T/200:T;

u_in = U * sin(omega*t);
u_r  = abs(U_R) * sin(omega*t + angle(U_R));
u_c  = abs(U_C) * sin(omega*t + angle(U_C));
u_l  = abs(U_L) * sin(omega*t + angle(U_L));

figure(1);
plot(t, u_in, t, u_r, t, u_c, t, u_l);
legend('u_{in}', 'u_r', 'u_c', 'u_l');
xlabel('t/s'); ylabel('u/V');
\end{lstlisting}

Phasor diagram of all voltages:
\begin{lstlisting}
figure(2);
compass(U  , 'k'); hold on;
compass(U_R, 'b'); hold on;
compass(U_C, 'g'); hold on;
compass(U_L, 'r'); hold off;
legend('U_{in}', 'U_R', 'U_C', 'U_L');
xlabel('Re U_k \rightarrow'); ylabel('Im U_k \rightarrow');
\end{lstlisting}

Bode plot of a transfer function:
\begin{lstlisting}
w = 2*pi*logspace(-1, 5); % Angular frequency (1/s)
H = ...;                  % Transfer function

figure(3);
subplot(2, 1, 1);              % Two rows, one column
semilogx(w, 20*log10(abs(H))); % Magnitude in dB
subplot(2, 1, 2);
semilogx(w, angle(H));         % Phase in rad
\end{lstlisting}

{\Large System Analysis using ODEs/Transfer functions}

Transfer function: $H(s) = \frac{Y(s)}{X(s)}$ with input $X(s)$ and output $Y(s)$

$y(t)\,\laplace\,Y(s)$, $\dot y(t)\,\laplace\,s\cdot Y(s)$, $\ddot y(t)\,\laplace\,s^2\cdot Y(s)$, $\dots$

Calculation of the range of the angular frequnecy for the Bode plot of the transfer function:
\begin{lstlisting}
omega_min = ...;
omega_max = ...;
w         = logspace(log10(omega_min), log10(omega_max));
\end{lstlisting}

{\Large State space}

Conversion of an ODE of $n$\textsuperscript{th} order to a system of $n$ ODEs of first order (input $u$, output $y$):\\
\[a_n\cdot y^{(n)}+a_{n-1}\cdot y^{(n-1)}+\dots+a_2\cdot\ddot y+a_1\cdot\dot y+a_0\cdot y=b_0\cdot u\]

Introduction of state space variables:\\
\[x_1 = y, x_2 = \dot y, x_3 = \ddot y, \dots, x_{n-1} = y^{(n-2)}, x_n=y^{(n-1)}\]
\[\Rightarrow \dot x_1 = x_2, \dot x_2 = x_3, \dots, \dot x_{n-1} = x_n\qquad \text{(}n-1\text{ equations)}\]

\[a_n\cdot\dot x_n+a_{n-1}\cdot x_n+a_{n-2}\cdot x_{n-1}+\dots+a_2\cdot x_3+a_1\cdot x_2+a_0\cdot x_1=b_0\cdot u\]
\[\Rightarrow \dot x_n = \frac{-a_{n-1}}{a_n}\cdot x_n-\frac{a_{n-2}}{a_n}\cdot x_{n-1}-\dots-\frac{a_2}{a_n}\cdot x_3-\frac{a_1}{a_n}\cdot x_2-\frac{a_0}{a_n}\cdot x_1+\frac{b_0}{a_n}\cdot u\qquad \text{(1 equation)}\]

Matrix representation (only for linear ODEs):
\[\frac{\mathrm{d}}{\mathrm{d}t}\begin{bmatrix}x_1\\\dots\\\dots\\x_n\end{bmatrix}=\underbrace{A}_\text{state matrix}\cdot\vec x+B\cdot\vec u\]
\[\vec y=\underbrace{C}_\text{output matrix}\cdot\vec x+D\cdot\vec u\]

{\Large Solving ODEs with Matlab}

Function-file \texttt{dgl.m} with the state space representation of the ODE in this form: $\dot{\vec x}=\vec f(\vec x)$:
\begin{lstlisting}
function xp = dgl(t, x, dummy, parameter1, parameter2, ...)

xp = zeros(n, 1); % xp: d/dt x, column vector
                  %  n: order of the ODE

xp(1) = ...;
xp(2) = ...;
% ...
xp(n) = ...;
\end{lstlisting}

Invocation of the solver:
\begin{lstlisting}
t_span = [0 t_max];         % Time interval for simulation
x0     = [x1_0, x2_0, ...]; % Initial values of the ODE

[t, x] = ode45('dgl', t_span, x0, [], parameter1, parameter2, ...);

% t: Time vector returned by the solver
% x: column 1: x1, column 2: x2, ...

figure(1); plot(t, x(:, 1)); % x1 plotted against t
figure(2); plot(t, x(:, 2)); % x2 plotted against t
\end{lstlisting}

Harmonic oscillator: $\ddot y + \delta\cdot\dot y + \omega_0^2\cdot y=\frac{1}{m}\cdot F_\text{ext}$

Damped free oscillation ($\delta>0$):
\begin{enumerate}
	\item $\delta<\omega_0$: Underdamped, $\omega=\sqrt{\omega_0^2-\delta^2}$
	\item $\delta=\omega_0$: Critically damped
	\item $\delta>\omega_0$: Overdamped
\end{enumerate}

Forced oscillation by excitation with an external force $F_\text{ext}(t)=A_0\cdot\sin(\omega_\mathrm{f}\cdot t)$:
\begin{enumerate}
	\item $\omega_\mathrm{f}\ll\omega_0$: System is in phase with the external excitation after some time.
	\item $\omega_\mathrm{f}\approx\omega_0$: Resonance, phase difference of $90^\circ$.
	\item $\omega_\mathrm{f}\gg\omega_0$: Phase difference of $180^\circ$
\end{enumerate}

{\Large Solving ODEs with Simulink}

Passing vectors from the Matlab workspace using the \emph{From Workspace} block, data format: Matrix with the time vector as the first column and data vector(s) as additional column(s):
\begin{lstlisting}
x = [1:10];         % Parameter as a row vector
t = 0:length(x)-1;  % Time as a row vector

param = [t.', x.']; % Simulink format for passing parameters
\end{lstlisting}

Transfer of vectors to the Matlab workspace using the \emph{To Workspace} block, Save Format: Array.

Invocation of a Simulink simulation from Matlab (Name of the Simulink model must not be identical to the name of the invoking Matlab m-file!):
\begin{lstlisting}
tout = sim('example_sim', time_vect);

% time_vect: defines temporal resolution and simulation interval
% tout     : internally used time vector
\end{lstlisting}

Graphical solution of ODEs using Simulink:
\begin{enumerate}
	\item ODE of $n$\textsuperscript{th} order: \[a_n\cdot y^{(n)}+a_{n-1}\cdot y^{(n-1)}+\dots+a_2\cdot\ddot y+a_1\cdot\dot y+a_0\cdot y=b_0\cdot u\]
	\item Solve for the highest derivative: \[y^{(n)}=\frac{-a_{n-1}}{a_n}\cdot y^{(n-1)}-\frac{a_{n-2}}{a_n}\cdot y^{(n-2)}-\dots-\frac{a_1}{a_n}\cdot\dot y-\frac{a_0}{a_n}\cdot y+\frac{b_0}{a_n}\cdot u\]
	\item Successive integration, starting with the highest derivate: \[y^{(n-1)}=\int y^{(n)},\,y^{(n-2)}=\int y^{(n-1)},\,\dots\,\dot y=\int\ddot y,\,y=\int\dot y\]
\item Simulink block diagram:
\end{enumerate}

{\Large Simulations using Simulink}

Avoiding algebraic loops: Insert \emph{Algebraic Constraint} block.
\end{document}
